\thispagestyle{plain} % Remove header
\addcontentsline{toc}{chapter}{General Conclusion}
\section*{General Conclusion}
Throughout the period of the internship within Incedo Services GmbH, we were committed to contributing to the growth of the company by making the Incedo Lead Generator software more successful.
The objective of this project was to add value to the existing software solution by, first of all, developing more features as well as fixing bugs and, second of all, by migrating the whole infrastructure of the monolithic application to use microservices instead.
The microservices in question were deployed to a self-managed instance of Kubernetes that we of course had to create and manage ourselves.
The whole setup also adheres to the best practices of DevOps such as automated pipelines and continuous monitoring.

The realization of this project has allowed us to face numerous constraints; mainly time constraints, technological constraints and also COVID-19 constraints that basically forced us to do this internship fully remotely. We nevertheless had a huge chance to broaden our knowledge pool in the field of IT in general and especially in the growing field of DevOps.
We were very lucky that the Incedo team encourages curiosity, learning and experimenting on the various new technologies which has motivated us to explore and develop our technical skills more and more.
Some of the technologies that we've learned thanks to this project are GitLab's advanced features, Ansible, Kubernetes, Helm, Prometheus, Grafana, Nginx, GraphQL and many more...

This internship has been a great opportunity to apply our theoretical and practical knowledge acquired at the Higher Institute of Technological Studies of Nabeul and to build much more on it.
Working at Incedo has truly allowed us to shine, especially considering how very pleasant it was to communicate with the team and we are extremely happy to have been recruited as permanent members of this wonderful family.

With that said, there are a couple of prospects looming on the horizon for improving the work we've done.
One thing that comes to mind is to create modern looking dashboards in Kibana for the microservices' logs that are already streamed to Elasticsearch in order to get more visibility on the errors, warnings or anything that's happening at a business logic layer inside the microservices for non IT people.

To conclude, it feels like we definitely succeeded our internship in the truest meaning of the word and we are ready to keep on building and enhancing ourselves as well as to contribute hugely to the industry.
