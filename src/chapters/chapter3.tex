\chapter{Analysis and specification of requirements}
\minitoc
\newpage

\setcounter{secnumdepth}{0} % Set the section counter to 0 so next section is not counted in toc
% ----------------------- Introduction ----------------------- %
\section{Introduction}
In this chapter, we are going to analyze the specification and the requirements of the application.
This chapter will describe the product's main features and their conception that should be met for the end result.

\setcounter{secnumdepth}{2} % Resume counting the sections for the toc with a depth of 2 (Sections and sub-sections)
% ----------------------- Requirements ----------------------- %
\section{Requirements}
Our requirements are divided into two parts:
\begin{itemize}
	\item The public client portal of the application or dashboard.
	\item The lead generating happening in the background.
\end{itemize}

\subsection{Application dashboard}

\subsubsection{Requirements}
Here is a global use case diagram that specifies the dashboard requirements:
\begin{figure}[H]
	\centering
	\makebox[\textwidth]{\includegraphics[width=10cm]{src/assets/diagrams/usecase.png}}
	\caption{Use case diagram}
	\label{fig:use-case-diagram}
\end{figure}

\subsubsection{Actors}
Mainly, there are 3 types of actors of the system:
\begin{itemize}
	\item \textbf{Super Admin}: The main administrator with full access permissions to the system.
	\item \textbf{Admin:} The administrator of the system with slightly less privileges than the Super Admin.
	\item \textbf{User:} The main user of the system, who benefits from the different features the system offers, usually a client representative.
\end{itemize}

\subsubsection{User management}
Here are the main requirements of user management for the different type of actors:
\begin{itemize}
	\item As a \textbf{Super Admin}, I can create, modify or delete users and admins accounts.
	\item As an \textbf{Admin}, I can create, modify or delete user's accounts (clients).
	\item As an \textbf{Admin}, I can add or remove credit from user's accounts (clients).
	\item As an \textbf{Admin}, I can view my user's accounts, their used and bought credits.
\end{itemize}

\subsubsection{Campaign management}
Here are the main requirements of campaign management for the different type of actors:
\begin{itemize}
	\item As a \textbf{User}, I can create my campaign specifying the LinkedIn account credentials to be used, the campaign list, salutation messages, invite messages, follow up messages and the number of invites per day.
	\item As a \textbf{User}, I can update my campaigns details such as the LinkedIn account credentials, the campaign list, salutation messages, invite messages, follow up messages and the number of invites per day.
	\item As a \textbf{User}, I can delete my campaigns.
	\item As a \textbf{User}, I can activate my campaigns.
	\item As a \textbf{User}, I can stop my campaigns.
	\item As a \textbf{User}, I can export information about the scraped leads as a CSV file.
\end{itemize}

\subsubsection{Campaign monitoring}
\begin{itemize}
	\item As a \textbf{Super Admin}, I can view the campaigns' records, on day by day basis or in a specific range of dates with information such as number of invites sent, number of follow ups sent and the number of withdrawn invites.
	\item As a \textbf{User}, I can view my campaigns' analytics with information such as requested, connected, replied leads, and conversion rates.
\end{itemize}

\subsection{Lead generating}

\subsubsection{Requirements}
The lead generating is the process of \textbf{scraping} the LinkedIn accounts and \textbf{generating} leads by sending them customized invites and follow ups messages through the LinkedIn UI depending on the user's campaign preferences.
So here is the requirements of both parts:

\subsubsection{Scraping leads}
\begin{itemize}
	\item Each day, the system generates a list of URLs to scrape from a campaign (search) list and saves them as tasks in the database where each list item corresponds to a lead's profile URL.
	\item Each day, the leads' information are scraped from the previously generated URLs. Each scraped URL corresponds to one lead and is saved as an entry in the database.
\end{itemize}

\subsubsection{Generating leads}
\begin{itemize}
	\item Each day, the system sends a certain number of invites to the scraped leads. The number is between a range the user specifies in the UI.
	\item The system sends out the first follow up message to leads that accepted the connection request but did not yet reply. The first follow up message is sent at the earliest 24 hours after the connection request has been sent.
	\item The second follow up message is sent out on the day after the first follow up message was sent if the lead did not reply yet.
\end{itemize}

\newpage

% ----------------------- Conception ----------------------- %
\section{Conception}
In this section, we will dive deep into the conception and the logic behind the parts of the application:

\subsection{Architecture}
\begin{figure}[H]
	\centering
	\makebox[\textwidth]{\includegraphics[width=17cm]{src/assets/diagrams/overview.png}}
	\caption{Overview of the ILG services}
	\label{fig:services-overview}
\end{figure}

\subsubsection{Microservices}
\begin{table}[H]
	\renewcommand{\arraystretch}{1.5}%
	\caption{All the application's Microservices}
	\centering
	\medskip
	\begin{tabularx}{1\textwidth} {
			| >{\hsize=.8\hsize\linewidth=\hsize\centering\arraybackslash}X
			| >{\hsize=1.75\hsize\linewidth=\hsize\justifying\arraybackslash}X
			| >{\hsize=0.45\hsize\linewidth=\hsize\centering\arraybackslash}X |}
		\hline
		\rowcolor{primary} \textbf{Name} & \noindent \textbf{Description}                                                                                                  & \textbf{Scalable} \\
		\hline
		\textbf {ilg-data}               & \noindent A GraphQL data access layer used to feed all needed data for other services from the PostgreSQL database.             & No                \\
		\hline
		\textbf {ilg-api}                & \noindent A RESTful api used to serve data to the React app dashboard.                                                          & No                \\
		\hline
		\textbf {ilg-front}              & \noindent A React app dashboard application served through NGINX.                                                               & No                \\
		\hline
		\textbf {ilg-scheduler}          & \noindent The service that's responsible for scheduling and orchestrating the tasks queues and notifications.                   & No                \\
		\hline
		\textbf {ilg-scraper}            & \noindent The service that's responsible for scraping leads from the campaign links generated from LinkedIn Sales Navigator.    & Yes               \\
		\hline
		\textbf {ilg-automation}         & \noindent The service that's responsible for handling LinkedIn accounts' inboxes, threads and for sending invites to the leads. & Yes               \\
		\hline
	\end{tabularx}
\end{table}

\subsubsection{Cloud Infrastructure}
Since we are deploying to a Kubernetes cluster, we decided to distribute our different microservices into their corresponding nodes or virtual machines depending on their shared components and scalability.
\begin{figure}[H]
	\centering
	\makebox[\textwidth]{\includegraphics[width=22cm]{src/assets/diagrams/cloud_infra.jpeg}}
	\caption{Cloud infrastructure of the application}
	\label{fig:cloud-infrastructure}
\end{figure}

\subsection{Tasks \& Queues}

\subsubsection*{\underline{In theory}}
Task queues allow services to perform work in the form of asynchronous tasks outside of a user request.
If an app needs to execute work in the background, it adds the tasks it needs to the task queues. The tasks are executed later, by worker services. The Task Queue service is designed for asynchronous work.
Every lead scraping or generating tasks are implemented as a task queue.
\begin{figure}[H]
	\centering
	\makebox[\textwidth]{\includegraphics[width=12cm]{src/assets/diagrams/task-queue.png}}
	\caption{Task queues}
	\label{fig:task-queue}
\end{figure}

\subsubsection*{\underline{Tasks queues types}}
\begin{table}[H]
	\renewcommand{\arraystretch}{1.5}%
	\caption{The application's tasks or jobs}
	\centering
	\medskip
	\begin{tabularx}{1\textwidth} {
			| >{\hsize=0.7\hsize\linewidth=\hsize\centering\arraybackslash}X
			| >{\hsize=1.3\hsize\linewidth=\hsize\justifying\arraybackslash}X |}
		\hline
		\rowcolor{primary} \textbf{Name} & \noindent \textbf{Description}                                                                  \\
		\hline
		\textbf {campaign-scrape}        & \noindent Scrapes a list of leads from a campaign search list or URL.                           \\
		\hline
		\textbf {lead-scrape}            & \noindent Scrapes the lead information from a URL and saves it in the database.                 \\
		\hline
		\textbf {handle-inbox}           & \noindent Opens the conversation inbox of leads and sends them follow up messages if necessary. \\
		\hline
		\textbf {send-invite}            & \noindent Sends a connection invite to leads through LinkedIn.                                  \\
		\hline
	\end{tabularx}
\end{table}

\subsubsection*{\underline{Scheduling tasks}}
Scheduling the different types of tasks is mainly done by the ilg-scheduler service on a daily basis according to a specific cron job we specify in code.
We use the database that's accessible by ilg-data, as the store of those task queues with all of their corresponding information and state.
The diagram below illustrates the different producers and consumers of the tasks in more details.
\begin{figure}[H]
	\centering
	\makebox[\textwidth]{\includegraphics[width=16cm]{src/assets/diagrams/producer_consumer.png}}
	\caption{Diagram detailing the producers and the consumers}
	\label{fig:producer-consumer-diagram}
\end{figure}

\subsection{Scraper \& Automation}
The ilg-scraper and ilg-automation services are responsible for consuming the tasks generated on a daily basis by the system.
Their main purpose is to generate leads for the various campaigns.

\subsubsection*{\underline{Scraper service flow}}
\begin{figure}[H]
	\centering
	\makebox[\textwidth]{\includegraphics[width=16cm]{src/assets/diagrams/campaign_scraper_sequence.png}}
	\caption{Campaign scraper sequence diagram}
	\label{fig:campaign-scraper-sequence-diagram}
\end{figure}
\begin{figure}[H]
	\centering
	\makebox[\textwidth]{\includegraphics[width=16cm]{src/assets/diagrams/lead_scraper_sequence.png}}
	\caption{Lead scraper sequence diagram}
	\label{fig:lead-scraper-sequence-diagram}
\end{figure}
\newpage

\subsubsection*{\underline{Automation service flow}}
\begin{figure}[H]
	\centering
	\makebox[\textwidth]{\includegraphics[width=16cm]{src/assets/diagrams/send_invite_sequence.png}}
	\caption{Diagram detailing the producers and the consumers}
	\label{fig:send-invite-sequence-diagram}
\end{figure}
\begin{figure}[H]
	\centering
	\makebox[\textwidth]{\includegraphics[width=16cm]{src/assets/diagrams/handle_inbox_sequence.png}}
	\caption{Diagram detailing the producers and the consumers}
	\label{fig:handle-inbox-sequence-diagram}
\end{figure}
\newpage

\subsection{Data Layers}
\subsubsection*{\underline{Schema or Entity Relationship Diagram (ERD)}}
\begin{figure}[H]
	\centering
	\makebox[\textwidth]{\includegraphics[width=16cm]{src/assets/diagrams/er.png}}
	\caption{Entity Relationship Diagram}
	\label{fig:entity-relationship-diagram}
\end{figure}
\newpage

\subsubsection*{\underline{Communication between services}}
For the intra-communication between the various services, two interfaces are provided:
\begin{itemize}
	\item \textbf{GraphQL API Layer:}
	      This interface is used by all the internal services of the application to access the data.
	      In other words, this layer is only reachable within the \textbf{internal private virtual network} and it is \textbf{not exposed publicly}.
	\item \textbf{RESTful API Layer:}
	      This interface is used by the frontend service to access the data. In other words, this layer is \textbf{exposed publicly}.
	      It is of course secured by a JWT token level authentication.
\end{itemize}

\setcounter{secnumdepth}{0} % Set the section counter to 0 so next section is not counted in toc
% ----------------------- Conclusion ----------------------- %
\section{Conclusion}
In this chapter, we discussed the main features of the application and its different layers and services.
In what follows, we will discuss the realization part with heavy focus on DevOps and the actual implementation of the new architecture.
