\chapter{Realization}
\newpage

\setcounter{secnumdepth}{0} % Set the section counter to 0 so next section is not counted in toc
% ----------------------- Introduction ----------------------- %
\section{Introduction}
In this chapter, we will finally discuss the long awaited realization.
This is the implementation phase where all of the work done in the previous chapters come into picture.
We will start with the architecture where we setup the necessary hardware and software tools.
Then we will present some of the difficulties we encountered.

\setcounter{secnumdepth}{2} % Resume counting the sections for the toc with a depth of 2 (Sections and sub-sections)
% ----------------------- Hardware environment ----------------------- %
\section{Hardware environment}
To make the most out of the ILG and for the sake of scalability, here's the setup we came up with.

\begin{table}[H]
    \renewcommand{\arraystretch}{1.5} % Padding
    \caption{Assignement of virtual machines to their respective domain names and services}
    \centering
    \medskip
    \begin{tabularx}{1\textwidth} {
            | >{\hsize=1.3\hsize\linewidth=\hsize\centering\arraybackslash}X
            | >{\hsize=1.4\hsize\linewidth=\hsize\centering\arraybackslash}X
            | >{\hsize=0.3\hsize\linewidth=\hsize\centering\arraybackslash}X |}
        \hline
        \rowcolor{primary} \textbf{Domain name} & \textbf{Services}             & \textbf{Size} \\
        \hline
        \textbf{data.ilg.incedo.net}            & ilg-data, ilg-api, postgresdb & RM            \\
        \hline
        \textbf{ps.ilg.incedo.net}              & ilg-front, ilg-scheduler      & RM            \\
        \hline
        \textbf{idr.ilg.incedo.net}             & A custom container registry   & RM            \\
        \hline
        \textbf{uspN.ilg.incedo.net}            & Upstream proxy server         & RM            \\
        \hline
        \textbf{slsN.ilg.incedo.net}            & ilg-scraper                   & RM            \\
        \hline
        \textbf{casN.ilg.incedo.net}            & ilg-automation                & RM            \\
        \hline
    \end{tabularx}
\end{table}
The two images below show the available VM sizes in IONOS.

\begin{figure}[H]
    \centering
    \makebox[\textwidth]{\includegraphics[width=\linewidth]{src/assets/images/vm-standard.jpg}}
    \caption{Standards VM sizes in IONOS}
    \label{fig:standard-vm-sizes}
\end{figure}

\begin{figure}[H]
    \centering
    \makebox[\textwidth]{\includegraphics[width=\linewidth]{src/assets/images/vm-ram-optimized.JPG}}
    \caption{RAM Optimized VM sizes in IONOS}
    \label{fig:ram-optimized-vm-sizes}
\end{figure}

We already discussed what each of the services does in the chapter <insert-section-name-and-page-here>.
Also note that this section is only relevant in production.

% ----------------------- Software environment ----------------------- %
\section{Software environment}
This part is reserved for the presentation of the software used in the realization of the application and includes but is not limited to; programming languages, frameworks, technologies, architectures, etc...

\medskip
\textbf {For more details, see Chapter 2: State of the art}

\begin{itemize}
    \item \textbf{Git:} \newline
          \begin{minipage}{\linewidth}
              \centering
              \includegraphics[width=2.5cm]{src/assets/logos/git_512x512.png}
              \captionof{figure}{Logo of Git}
          \end{minipage}
    \item \textbf{Docker:} \newline
          \begin{minipage}{\linewidth}
              \centering
              \includegraphics[width=4cm]{src/assets/logos/docker_512x512.png}
              \captionof{figure}{Logo of Docker}
          \end{minipage}
    \item \textbf{Playwright:} \newline
          \begin{minipage}{\linewidth}
              \centering
              \includegraphics[width=4.5cm]{src/assets/logos/playwright_512x512.png}
              \captionof{figure}{Logo of Playwright}
          \end{minipage}

          \newpage
    \item \textbf{NestJS:} \newline A framework for building efficient, scalable Node.js web applications. \newline
          \begin{minipage}{\linewidth}
              \centering
              \includegraphics[width=3.7cm]{src/assets/logos/nestjs_512x512.png}
              \captionof{figure}{Logo of NestJS}
          \end{minipage}
    \item \textbf{ReactJS:} \newline A front-end javascript library that's often used as a framework. \newline
          \begin{minipage}{\linewidth}
              \centering
              \includegraphics[width=3.7cm]{src/assets/logos/react_512x512.png}
              \captionof{figure}{Logo of ReactJS}
          \end{minipage}
    \item \textbf{MySQL:} \newline
          \begin{minipage}{\linewidth}
              \centering
              \includegraphics[width=5cm]{src/assets/logos/mysql.png}
              \captionof{figure}{Logo of ReactJS}
          \end{minipage}

          \newpage
    \item \textbf{Sequelize:} \newline A front-end javascript library that's often used as a framework. \newline
          \begin{minipage}{\linewidth}
              \centering
              \includegraphics[width=3.7cm]{src/assets/logos/react_512x512.png}
              \captionof{figure}{Logo of MySQL}
          \end{minipage}
    \item \textbf{GraphQL:} \newline A front-end javascript library that's often used as a framework. \newline
          \begin{minipage}{\linewidth}
              \centering
              \includegraphics[width=3.7cm]{src/assets/logos/react_512x512.png}
              \captionof{figure}{Logo of ReactJS}
          \end{minipage}
    \item \textbf{GitLab:} \newline All of GitLab's tools such as groups, labels and CI/CD for the creation of pipelines. \newline
          \begin{minipage}{\linewidth}
              \centering
              \includegraphics[width=3.5cm]{src/assets/logos/makefile_512x512.png}
              \captionof{figure}{Logo of Makefile}
          \end{minipage}

          \newpage
    \item \textbf{Nginx:} \newline A multi-purpose web server can also be used as reverse proxy, load balancer, mail proxy and HTTP cache. \newline
          \begin{minipage}{\linewidth}
              \centering
              \includegraphics[width=3.5cm]{src/assets/logos/nginx_512x512.png}
              \captionof{figure}{Logo of Makefile}
          \end{minipage}
    \item \textbf{Renovate:} \newline GNU Make is used extensively throughout the application to save useful commands for the ease of maintainability. \newline
          \begin{minipage}{\linewidth}
              \centering
              \includegraphics[width=3.5cm]{src/assets/logos/makefile_512x512.png}
              \captionof{figure}{Logo of Makefile}
          \end{minipage}
    \item \textbf{Make:} \newline GNU Make is used extensively throughout the application to save useful commands for the ease of maintainability. \newline
          \begin{minipage}{\linewidth}
              \centering
              \includegraphics[width=3.5cm]{src/assets/logos/makefile_512x512.png}
              \captionof{figure}{Logo of Makefile}
          \end{minipage}

          \newpage
\end{itemize}

% ----------------------- Difficulties encountered ----------------------- %
\section{Difficulties encountered}
This not persay a technical difficulty as it is more of a limitation from the side of the IONOS cloud provider the organization relies on.
To have a full DevOps approach, even the infrastructure should be managed in code -also known as IAC (Infrastructure As Code)- using tools such as Terraform.
However, the cloudpanel version that we were using does not provide that functionality.
Therefore we had to scrape off the idea which was a bit unfortunate.

\setcounter{secnumdepth}{0} % Set the section counter to 0 so next section is not counted in toc
% ----------------------- Conclusion ----------------------- %
\section{Conclusion}
A successfull project comes from a successfull development environment as well as a clean production environment.
For this reason, we spent quite a good amount of time setting up the infrastructure in a clean and scalable way.
