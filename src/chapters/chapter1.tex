\chapter{Context of the work}
\minitoc
\newpage

\setcounter{secnumdepth}{0} % Set the section counter to 0 so next section is not counted in toc
% ----------------------- Introduction ----------------------- %
\section{Introduction}
This chapter introduces the general context of this report.
We start by presenting the frame of the project as well as the host company.
Then comes the enumeration of the problems which led to the realization of the project.
We wrap it up by defining the methodology we’ve followed to carry out our work.

\setcounter{secnumdepth}{2} % Resume counting the sections for the toc with a depth of 2 (Sections and sub-sections)
% ----------------------- General framework of the internship ----------------------- %
\section{General framework of the internship}
This project was carried out within the frame of obtaining a bachelor’s degree in Computer Science at the Higher Institute of Technological Studies of Nabeul (ISETN).
The internship took place fully remotely at Incedo Services GmbH for five months starting from the 26th of January 2022 to the 30th of June 2022 with the purpose of further developing an existing software solution of the company as well as migrating its infrastructure for scalability and to adhere to modern DevOps principles.

% ----------------------- Company overview ----------------------- %
\section{Company overview}
This section introduces the host company {\bf Incedo Services GmbH} as well as the services it offers.
\subsection{About Incedo}
"We are a young software development and consulting firm located in Stuttgart.
We help our clients to develop exciting digital products and solutions and we also love to bring our own ideas into life from time to time." \cite{about-incedo}
\begin{figure}[H]
    \centering
    \makebox[\textwidth]{\includegraphics[width=12cm]{src/assets/images/incedo-logo_bg-trans-white_rgb.png}}
    \caption{Logo of Incedo Services GmbH}
    \label{fig:logo-of-incedo}
\end{figure}

\subsection{Incedo's services}
\subsubsection*{\underline{Consulting}}
Incedo helps its clients overcome two main challenges:
\begin{itemize}
    \item Develop a product that somebody actually needs and that creates enough value to form a profitable business case.
    \item Ensuring that (large) IT-projects are finished in time \& budget with a result that satisfies the actual users/clients of the developed software solution.
\end{itemize}

\subsubsection*{\underline{Development}}
Incedo develops software for clients, as well as for the company itself.
\begin{figure}[H]
    \centering
    \makebox[\textwidth]{\includegraphics[width=\linewidth]{src/assets/images/selection-of-clients.JPG}}
    \caption{Selection of Incedo clients}
    \label{fig:selection-of-incedo-clients}
\end{figure}

% ----------------------- Stating the problem ----------------------- %
\section{Stating the problem}
Incedo -having ambitious goals to grow over the next 2 to 4 years- wants to win new clients and strategically develop the existing ones. Winning new clients starts with generating new leads. Previously, Incedo has worked with an Austrian start-up (motion group) that provides automated lead generation through LinkedIn at the cost of 0.85 € per requested contact. Although one big project was closed and several leads were generated, Incedo started using the LinkedIn Sales Navigator with a more targeted approach, but nevertheless wanted to automate the approach and increase its outreach.
Having launched the first version of the \glsxtrshort{ilg} as a \glsxtrshort{saas}, Incedo was satisfied for a while.
Over the time however, as more and more clients were interested in the \glsxtrshort{ilg}, the current architecture couldn’t handle the load properly. Therefore, it was our task to re-design and implement a new scalable architecture instead of the old one.

% ----------------------- Assessment of the case ----------------------- %
\section{Assessment of the case}
\subsection{Describing the work procedure}
The work on any project must first of all be preceded by a thorough study of the existing ones which undermines the strengths and weaknesses of the current system, as well as the business decisions that should be taken into account during the conception as well as the realization.
\subsection{Criticizing the current state}
After studying the existing, we can determine its limitations:
\begin{itemize}
    \item Bugs always tend to happen whenever the LinkedIn website changes (due to scraping).
    \item Since cron jobs are running for the whole day, bugs are hard to respond to fast enough because we can only deploy once at the end of day.
    \item It is hard to test the whole workflow because of how the app works (looking for certain changes in the LinkedIn interface after certain buttons are clicked for example) meaning that the dev environment is lacking.
    \item It cannot scale well enough since the whole monolithic application is deployed on a single server.
    \item It is very hard to read the logs of the application since we have to SSH into the \glsxtrshort{vm} every time and use the correct grep command.
\end{itemize}
\subsection{Proposed solution}
The solution to these problems is refactoring the whole application to separate sub-applications or microservices where the automation and scraping processes can be scaled independently of the other parts of the application.
This way, it will be easier to maintain and scale the different codebases as well as respond faster to bugs and know exactly what caused them in the first place.

\newpage

% ----------------------- Development Methodology ----------------------- %
\section{Development Methodology}
\subsection{Agile methodology}
Agile is a structured and iterative approach to project management and product development.
It recognizes the volatility of product development, and provides a methodology for self-organizing teams to respond to change without going off the rails.

\subsection{Scrum methodology}
Scrum teams commit to completing an increment of work, which is potentially shippable, through set intervals called sprints.
Their goal is to create learning loops to quickly gather and integrate customer feedback.
Scrum teams adopt specific roles, create special artifacts, and hold regular ceremonies to keep things moving forward.

\subsection{Kanban methodology}
Kanban is all about visualizing your work, limiting work in progress, and maximizing efficiency (or flow).
Kanban teams focus on reducing the time a project takes from start by using a Kanban board to continuously improve their flow of work.
To explain more in details, Kanban is based on a continuous workflow structure that keeps teams nimble and ready to adapt to changing priorities.
Work items —represented by cards— are organized on the board where they flow from one stage of the workflow or column to the next.
Common workflow stages are To Do, In Progress, In Review, and Done.

\begin{figure}[H]
    \centering
    \makebox[\textwidth]{\includegraphics[width=8cm]{src/assets/images/simple-kanban-board.jpg}}
    \caption{Simple Kanban Board}
    \label{fig:simple-kanban-board}
\end{figure}

\subsection{The choice for ILG}
Since this project is very susceptible to changes from outside (LinkedIn), Kanban offers the most flexibility in comparison to Scrum so that’s why we went with it instead.

\begin{figure}[H]
    \centering
    \makebox[\textwidth]{\includegraphics[width=\linewidth]{src/assets/images/kanban-board.JPG}}
    \caption{ILG project Kanban board}
    \label{fig:ilg-project-kanban-board}
\end{figure}

\subsection{Unified Modeling Language}
The Unified Modeling Language (UML) is a general-purpose, developmental, modeling language in the field of software engineering that is intended to provide a standard way to visualize the design of a system. In our case, we used UML to design the top level view of the systems therefore we only used the Use Case and Sequence diagrams.

\setcounter{secnumdepth}{0} % Set the section counter to 0 so next section is not counted in toc
% ----------------------- Conclusion ----------------------- %
\section{Conclusion}
In this chapter, we presented not only the host organization but also the general context of the project and why it's needed.
We then criticized the current state of the \glsxtrshort{ilg} application and discussed the possible development methodologies.
Lastly, we picked the most ideal choice for us which is Kanban.
