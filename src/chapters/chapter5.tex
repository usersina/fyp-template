\chapter{Realization: Environment}
\minitoc
\newpage

\setcounter{secnumdepth}{0} % Set the section counter to 0 so next section is not counted in toc
% ----------------------- Introduction ----------------------- %
\section{Introduction}
A project can always look simple from the outside, however if we dive deeply into how most software is made, we can see that it's not that simple of a process.
Therefore, this chapter will list all of the technologies that we've used and also some of the major technical difficulties we've come in contact with during the realization of our project.

\setcounter{secnumdepth}{2} % Resume counting the sections for the toc with a depth of 2 (Sections and sub-sections)
% ----------------------- Technologies ----------------------- %
\section{Technologies}
This part is reserved for the presentation of all of the software used in the realization of the project and includes but is not limited to; programming languages, frameworks, technologies, etc...

\medskip
For a comparative analysis on some of our choices, see \textbf {Chapter 2: State of the art}.

\begin{itemize}
    \item \textbf{Git:} \newline
          \begin{minipage}{\linewidth}
              \centering
              \includegraphics[width=2.5cm]{src/assets/logos/git_512x512.png}
              \captionof{figure}{Logo of Git}
          \end{minipage}
    \item \textbf{Docker:} \newline
          \begin{minipage}{\linewidth}
              \centering
              \includegraphics[width=4cm]{src/assets/logos/docker_512x512.png}
              \captionof{figure}{Logo of Docker}
          \end{minipage}
    \item \textbf{Playwright:} \newline
          \begin{minipage}{\linewidth}
              \centering
              \includegraphics[width=4.5cm]{src/assets/logos/playwright_512x512.png}
              \captionof{figure}{Logo of Playwright}
          \end{minipage}

          \newpage
    \item \textbf{NestJS:} \newline \cite{nestjs} A framework for building efficient, scalable Node.js web applications. \newline
          \begin{minipage}{\linewidth}
              \centering
              \includegraphics[width=3.7cm]{src/assets/logos/nestjs_512x512.png}
              \captionof{figure}{Logo of NestJS}
          \end{minipage}
    \item \textbf{ReactJS:} \newline A front-end JavaScript library that's often considered as a framework. \newline
          \begin{minipage}{\linewidth}
              \centering
              \includegraphics[width=3.7cm]{src/assets/logos/react_512x512.png}
              \captionof{figure}{Logo of ReactJS}
          \end{minipage}
    \item \textbf{PostgreSQL:} \newline \newline
          \begin{minipage}{\linewidth}
              \centering
              \includegraphics[width=8cm]{src/assets/logos/postgresql.png}
              \captionof{figure}{Logo of PostgreSQL}
          \end{minipage}

          \newpage
    \item \textbf{Sequelize:} \newline \cite{sequelize} A modern TypeScript and Node.js ORM for SQL databases. \newline
          \begin{minipage}{\linewidth}
              \centering
              \includegraphics[width=5cm]{src/assets/logos/sequelize_512x512.png}
              \captionof{figure}{Logo of Sequelize}
          \end{minipage}
    \item \textbf{GraphQL:} \newline \cite{graphql} A query language for APIs and a runtime for fulfilling those queries with existing data. \newline
          \begin{minipage}{\linewidth}
              \centering
              \includegraphics[width=3.7cm]{src/assets/logos/graphql_512x512.png}
              \captionof{figure}{Logo of GraphQL}
          \end{minipage}
    \item \textbf{GitLab:} \newline A DevOps software that allows secure collaboration and operations in a single application. \newline
          \begin{minipage}{\linewidth}
              \centering
              \includegraphics[width=3.5cm]{src/assets/logos/gitlab_512x512.png}
              \captionof{figure}{Logo of GitLab}
          \end{minipage}

          \newpage
    \item \textbf{Nginx:} \newline \cite{nginx} A multi-purpose web server can also be used as reverse proxy, load balancer, mail proxy and HTTP cache. \newline \newline
          \begin{minipage}{\linewidth}
              \centering
              \includegraphics[width=3.4cm]{src/assets/logos/nginx_512x512.png}
              \captionof{figure}{Logo of Nginx}
          \end{minipage}
    \item \textbf{Renovate:} \newline A software that provides automatic dependency updates with support for multiple languages. \newline
          \begin{minipage}{\linewidth}
              \centering
              \includegraphics[width=3cm]{src/assets/logos/renovate_200x200.png}
              \captionof{figure}{Logo of Renovate}
          \end{minipage}
    \item \textbf{Docker compose:} \newline A helper tool to run applications with multiple Docker containers using a yaml format. Often used in development. \newline \newline
          \begin{minipage}{\linewidth}
              \centering
              \includegraphics[width=4cm]{src/assets/logos/docker-compose_667x667.png}
              \captionof{figure}{Logo of Docker Compose}
          \end{minipage}

          \newpage
    \item \textbf{Visual Studio Code:} \newline A code editor that's used as an entire development environment with the help of extensions. \newline
          \begin{minipage}{\linewidth}
              \centering
              \includegraphics[width=2.5cm]{src/assets/logos/vscode_512x512.png}
              \captionof{figure}{Logo of VSCode}
          \end{minipage}
    \item \textbf{Linux:} \newline A Unix-like operating system for common daily use and more commonly for servers. \newline
          \begin{minipage}{\linewidth}
              \centering
              \includegraphics[width=3cm]{src/assets/logos/linux_512x512.png}
              \captionof{figure}{Logo of Linux}
          \end{minipage}
    \item \textbf{Make:} \newline GNU Make is used extensively throughout the application to save useful commands for the ease of maintainability. \newline
          \begin{minipage}{\linewidth}
              \centering
              \includegraphics[width=3cm]{src/assets/logos/makefile_512x512.png}
              \captionof{figure}{Logo of Makefile}
          \end{minipage}

          \newpage
    \item \textbf{MicroK8s:} \newline \cite{microk8s} MicroK8s is a simple production-grade conformant K8s. It is Lightweight and focused. \newline \newline
          \begin{minipage}{\linewidth}
              \centering
              \includegraphics[width=3.8cm]{src/assets/logos/microk8s.png}
              \captionof{figure}{Logo of MicroK8s}
          \end{minipage}
    \item \textbf{Kubernetes:} \newline The most powerful tool for managing containerized workloads in the cloud. \newline
          \begin{minipage}{\linewidth}
              \centering
              \includegraphics[width=3.5cm]{src/assets/logos/kubernetes_512x512.png}
              \captionof{figure}{Logo of Kubernetes}
          \end{minipage}
    \item \textbf{Ansible:} \newline An open-source software for provisioning, configuring and managing infrastructure. \newline
          \begin{minipage}{\linewidth}
              \centering
              \includegraphics[width=3.5cm]{src/assets/logos/ansible_200x200.png}
              \captionof{figure}{Logo of Ansible}
          \end{minipage}

          \newpage
    \item \textbf{Squid Proxy:} \newline \cite{squid} Squid is a caching proxy for the Web that supports HTTP, HTTPS, FTP and more. \newline
          \begin{minipage}{\linewidth}
              \centering
              \includegraphics[width=6cm]{src/assets/logos/squid-proxy.png}
              \captionof{figure}{Logo of Squid Proxy}
          \end{minipage}
    \item \textbf{GitLab agent for Kubernetes:} \newline \cite{kas-blog} A secure and reliable way to attach a Kubernetes cluster to GitLab. \newline
          \begin{minipage}{\linewidth}
              \centering
              \includegraphics[width=3.8cm]{src/assets/logos/gitlab-kubernetes-agent_512x512.png}
              \captionof{figure}{Logo of GitLab Agent for Kubernetes}
          \end{minipage}
    \item \textbf{Helm:} \newline Helm is the most popular package manager for Kubernetes. \newline
          \begin{minipage}{\linewidth}
              \centering
              \includegraphics[width=4cm]{src/assets/logos/helm_512x512.png}
              \captionof{figure}{Logo of Helm}
          \end{minipage}

          \newpage
    \item \textbf{Prometheus:} \newline Prometheus is a software application used for monitoring and alerting. \newline
          \begin{minipage}{\linewidth}
              \centering
              \includegraphics[width=4cm]{src/assets/logos/prometheus_512x512.png}
              \captionof{figure}{Logo of Prometheus}
          \end{minipage}
    \item \textbf{Grafana:} \newline Grafana is an open source solution for monitoring and analytics. \newline \newline
          \begin{minipage}{\linewidth}
              \centering
              \includegraphics[width=3.5cm]{src/assets/logos/grafana.png}
              \captionof{figure}{Logo of Grafana}
          \end{minipage}
    \item \textbf{LaTeX:} \newline \cite{latex-project} A high-quality document preparation and typesetting system for technical grade documents. \newline \newline
          \begin{minipage}{\linewidth}
              \centering
              \includegraphics[width=4cm]{src/assets/logos/latex_200x200.png}
              \captionof{figure}{Logo of The LaTeX Project}
          \end{minipage}

          \newpage
    \item \textbf{Elasticsearch:} \newline Elasticsearch is a search engine based on the Lucene library. Is it often used as part of a stack alongside Kibana and other tools. \newline
          \begin{minipage}{\linewidth}
              \centering
              \includegraphics[width=7cm]{src/assets/logos/elasticsearch-logo.png}
              \captionof{figure}{Logo of elasticsearch}
          \end{minipage}
    \item \textbf{Kibana:} \newline \cite{kibana} Kibana is an free and open frontend application that sits on top of the Elastic Stack, providing search and data visualization capabilities for data indexed in Elasticsearch. \newline
          \begin{minipage}{\linewidth}
              \centering
              \includegraphics[width=6.8cm]{src/assets/logos/kibana-logo.jpg}
              \captionof{figure}{Logo of Kibana}
          \end{minipage}
    \item \textbf{Fluentd:} \newline \cite{fluentd} Fluentd is an open source data collector, which lets you unify the data collection and consumption for a better use and understanding of data. \newline
          \begin{minipage}{\linewidth}
              \centering
              \includegraphics[width=4cm]{src/assets/logos/fluentd-logo.png}
              \captionof{figure}{Logo of Fluentd}
          \end{minipage}

          \newpage
    \item \textbf{Microsoft Teams:} \newline Microsoft Teams is a proprietary business communication platform developed by Microsoft. We use it to collaborate and share information. \newline
          \begin{minipage}{\linewidth}
              \centering
              \includegraphics[width=3.2cm]{src/assets/logos/ms-teams-logo.png}
              \captionof{figure}{Logo of Microsoft Teams}
          \end{minipage}
    \item \textbf{Asana:} \newline Asana is a work management platform available on web and and mobile. It is used internally by Incedo for various work and project management. \newline
          \begin{minipage}{\linewidth}
              \centering
              \includegraphics[width=4.5cm]{src/assets/logos/asana-logo.png}
              \captionof{figure}{Logo of Asana}
          \end{minipage}
    \item \textbf{Signal:} \newline Signal is a cross-platform centralized e2e encrypted instant messaging service. It is used by the team to share sensitive information. \newline
          \begin{minipage}{\linewidth}
              \centering
              \includegraphics[width=6cm]{src/assets/logos/signal-logo.png}
              \captionof{figure}{Logo of Signal}
          \end{minipage}

          \newpage
\end{itemize}

% ----------------------- Difficulties encountered ----------------------- %
\section{Difficulties encountered}
\subsection{Infrastructure as code}
This not per say a technical difficulty as it is more of a limitation from the side of Ionos cloudpanel; the service of IONOS Cloud that we use the provision the needed resources.
To have a full DevOps approach, even the infrastructure should be managed in code -also known as \glsxtrfull{iac}- using tools such as Terraform.
However, Ionos cloudpanel does not provide that functionality.
Therefore we had to scrape off the idea and simply create the virtual machines from the UI every time we need them.

\subsection{Stateful Kubernetes}
Another issue we had is data persistence in Kubernetes.
Kubernetes is -by design- used for stateless applications.
However, since we're using an Elasticsearch database, we definitely needed a way to persist our data across system reboots or crashes.

First of all, we used IONOS -our cloud provider- to create the physical Block Storage in question that we used as the data source.
We started with an SSD of 40 Gigabytes then scaled it up to 80 Gigs.

\begin{figure}[H]
    \centering
    \makebox[\textwidth]{\includegraphics[width=\linewidth]{src/assets/images/ionos-block-storage.JPG}}
    \caption{Creating a block storage in IONOS}
    \label{fig:ionos-block-storage}
\end{figure}

\newpage

Secondly, we had to logically mount the previously created block storage to a directory such as \textbf{/mnt/block} inside the controller virtual machine.

From there, we create a self-managed NFS Server on the host system that we export to the private network or in other words to all of the cluster nodes.

Now, for the stateful cluster components to use the NFS Server, we needed to setup an NFS client provisioner inside the Kubernetes cluster itself. This was as simple as using the \textbf{NFS Subdir External Provisioner} Helm Chart.

Lastly, our cluster components are now allowed to provision persistent and dynamic storage resources whenever they need it.

\subsection{Automatic Testing}
One of the keys to writing maintainable and error-prone software is to write tests across all the codebase to ensure no breaking changes happen when adding new pieces.
It also makes debugging way more manageable later on.
In our case however, it's challenging to write end-to-end or integration tests because most of the scraping and automation logic is dependent on interacting with LinkedIn via the browser.
And since LinkedIn sets limits on how many leads profiles we can view per day, mocking and running those tests will make our scraping and automation more detectable.
We still nonetheless have some parts of the codebases automatically tested such as modules integrations, helpers and authentication endpoints.

\begin{figure}[H]
    \centering
    \makebox[\textwidth]{\includegraphics[width=\linewidth]{src/assets/app-screenshots/ilg-backend-tests.png}}
    \caption{ilg-scraper tests}
    \label{fig:ilg-scraper-tests}
\end{figure}

\begin{figure}[H]
    \centering
    \makebox[\textwidth]{\includegraphics[width=\linewidth]{src/assets/app-screenshots/ilg-api-tests.png}}
    \caption{ilg-api tests}
    \label{fig:ilg-api-tests}
\end{figure}


\setcounter{secnumdepth}{0} % Set the section counter to 0 so next section is not counted in toc
% ----------------------- Conclusion ----------------------- %
\section{Conclusion}
In this chapter, we listed all of the technology stack that we used for ILG as well as some of the most weighty issues we've encountered.
We also listed some our solutions for the ones we could solve following the best practices of software development.
